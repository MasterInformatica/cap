\documentclass[10pt]{beamer}

\usetheme{m}

\usepackage{booktabs}
\usepackage[scale=2]{ccicons}

%\usepackage{pgfplots}
%\usepgfplotslibrary{dateplot}

\usepackage[latin1]{inputenc}
\usepackage[spanish]{babel}

\usepackage{listings}


\title{}
\subtitle{Ejemplo: openMP}
\date{}
\author{Luis Mar�a Costero Valero\\Jes�s Javier Domenech Arellano\\Hristo
  Ivanov Ivanov}
\institute{11 Enero 2016}

\titlegraphic{\hfill\includegraphics[scale=0.2]{images/logo.png}}


\metroset{titleformat=regular}
\metroset{inner/sectiontitleformat=regular}
\metroset{outer/frametitleformat=regular}
\metroset{block=fill}


%== Estilos propios de lstlisting
\lstset{%
  %backgroundcolor=\color{yellow!20},%
    basicstyle=\tiny\ttfamily,%
    numbers=left, numberstyle=\tiny, stepnumber=1, numbersep=5pt,%
    frame=single%
    }%

% Add your keywords here, and have this in a separate file
% and include it in your preamble
\lstset{emph={%  
    procedure, end, to, do, for, pragma, omp, parallel%
    },emphstyle={\color{black}\bfseries\underline}%
}%



\begin{document}

\maketitle



%%======= INDICE ========================================
%\begin{frame}
%  \frametitle{�ndice}
%  \setbeamertemplate{section in toc}[sections numbered]
%  \tableofcontents[hideallsubsections]
%\end{frame}

%\section{Introduction} %=== Esrto genera una p�gina de secci�n.



\begin{frame}[fragile,fragile]
  \frametitle{Cholesky}
\begin{lstlisting}
procedure CHOLESKY(A)
int i, j, k;
for k:= 0 to n-1 do
   A[k,k] := sqrt( A [k,k] );    /* Obtener la raiz cuadrada de la diagonal */
   for j := k+1 to n-1 do
      A[k,j] := A[k,j]/A[k,k];    /* Fase de divisi�n */
   end for
   for i := k+1 to n-1 do
      for j := i to n-1 do
         A[i,j] := A[i,j] - A[k,i] * A[k,j];    /* Fase de eliminaci�n */
      end for
   end for
end for
\end{lstlisting}

Dada una matriz cuadrada $A$ definida positiva, obtener otra matriz $C$
triangular inferior, tal que:
$$ A = C * C^t $$

\begin{itemize}
\item Resoluci�n de sistemas mediante factorizaci�n de la matriz.
\item Complejidad $O(N^3 / 3)$.
\item F�cilmente paralelizable.
\end{itemize}


\end{frame}






\begin{frame}
  \frametitle{Pruebas realizadas}
\end{frame}

\begin{frame}[fragile]
  \frametitle{Cholesky usando OPenMP}
  
\begin{lstlisting}
void chol_using_openmp(const Matrix A, Matrix U){
   unsigned int i, j, k; 
   unsigned int size = A.num_rows * A.num_columns;

   //#pragma omp parallel for 
   for (i = 0; i < size; i ++)
      U.elements[i] = A.elements[i];
   for(k = 0; k < U.num_rows; k++){
      U.elements[k * U.num_rows + k] = sqrt(U.elements[k * U.num_rows + k]);
      // Division step
      #pragma omp parallel for
      for(j = (k + 1); j < U.num_rows; j++)
         U.elements[k * U.num_rows + j] /= U.elements[k * U.num_rows + k]; 
      // Elimination step
      float * elems = U.elements;
      omp_set_num_threads(NUM_OMPTHREADS); // Set the number of threads		
      #pragma omp parallel for private(i,j)
      for(i = (k + 1); i < U.num_rows; i++){
         for(j = i; j < U.num_rows; j++){
            U.elements[i * U.num_rows + j] -= U.elements[k * U.num_rows + i] * U.elements[k * U.num_rows + j];
         }
      }
   }

   for(i = 0; i < U.num_rows; i++)
      for(j = 0; j < i; j++)
         U.elements[i * U.num_rows + j] = 0.0;
}
\end{lstlisting}
\end{frame}

%======= BIBLIOGRAF�A =======
\begin{frame}
  \frametitle{Bibliograf�a}
  
  \begin{enumerate}
  \item \textbf{Wikipedia:}\\\url{https://en.wikipedia.org/}
  \end{enumerate}
\end{frame}

\end{document}
