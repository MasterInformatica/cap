\documentclass[10pt]{beamer}

\usetheme{m}

\usepackage{booktabs}
\usepackage[scale=2]{ccicons}

%\usepackage{pgfplots}
%\usepgfplotslibrary{dateplot}

\usepackage[latin1]{inputenc}
\usepackage[spanish]{babel}

\usepackage{listings}
\usepackage{pgfplots}

\usepackage{amsmath}

\title{Maui}
\subtitle{Cluster Scheduler}
\date{}
\author{Luis Mar�a Costero Valero\\Jes�s Javier Dom�nech Arellano\\Hristo
  Ivanov Ivanov}
\institute{26 Enero 2016}

%\titlegraphic{\hfill\includegraphics[scale=0.2]{HPFF_logo.jpg}}

\definecolor{bgg}{HTML}{FBFBFB}
\def\gcolor{bgg}    % while presenting
%\def\gcolor{black} % while developing

\def\tikzpicdim{
  \draw[step=0.1cm, color=\gcolor] (0,-1) grid (12,7);
  \draw[step=1cm, color=\gcolor] (0,-1) grid (12,7);
}

\let\tikzpicdimlarge\tikzpicdim

\def\myurl{\hfil\penalty 100 \hfilneg \hbox}

\metroset{titleformat=regular}
\metroset{inner/sectiontitleformat=regular}
\metroset{outer/frametitleformat=regular}
\metroset{block=fill}

\lstset{%
  %backgroundcolor=\color{yellow!20},%
    basicstyle=\tiny\ttfamily,%
    numbers=left, numberstyle=\tiny, stepnumber=1, numbersep=5pt,%
    frame=l%
    }%


\begin{document}

\maketitle

%%======= INDICE ========================================
%\begin{frame}
%  \frametitle{�ndice}
%  \setbeamertemplate{section in toc}[sections numbered]
%  \tableofcontents[hideallsubsections]
%\end{frame}

%\section{Introduction} %=== Esrto genera una p�gina de secci�n.

\begin{frame}
  \frametitle{�Qu� es?}
Maui es un planificador de tareas para clusters.\\
Se desarroll� a mediados de los 90.\\

Las pol�ticas de planificaci�n de Maui abarcan:
\begin{itemize}
\item Prioridad din�mica.
\item Reservas.
\item fairShare \emph{\footnotesize(Compartici�n justa)}
\end{itemize}
\end{frame}

\begin{frame}
  \frametitle{Funcionalidades}
  \begin{enumerate}
  \item \textbf{Control de Tr�fico:} Gestiona las tareas para que no interfieran. 
  \item \textbf{Centrado en Objetivos:} Configuraci�n de cluster seg�n la tarea
  \item \textbf{Optimizaciones:} Supeditado a los dos puntos
    anteriores, el planificador intenta aprovechar los recursos del cluster.
  \end{enumerate}
\end{frame}

\begin{frame}
  \frametitle{Tipos de Nodos}
  SHARED
	Tasks from any combination of jobs may utilize available resources
SINGLEUSER
	Tasks from any jobs owned by the same user may utilize available resources
SINGLEJOB
	Tasks from a single job may utilize available resources
SINGLETASK
	A single task from a single job may run on the node
\end{frame}


\begin{frame}
  \frametitle{Job}
  
\textbf{Job: } La unidad de trabajo de Maui (lo que va a ejecutar).\\
Un \emph{job} est� formado a cu vez por:

\begin{enumerate}
\item Due�o del trabajo.
\item Varios \textbf{Requisitos} (cada uno con sus limitacioens de recursos).
\item Tiempo m�ximo de ejecuci�n.
\end{enumerate}\\

Un \textbf{Requisito} est� formado por:
\begin{itemize}
\item Definici�n de la tarea.
\item L�mite de recursos: Procesadores, memoria, swap, disk.
\item Task count: N� de instancias de la tarea a ejecutar en el requisito.
\item Task list: Lista de nodos en los que la tarea tiene que ser
  ejecutada.
\item Recursos estad�sticos: Qu� aspectos de la tarea monitorear.
\end{itemize}

% \begin{block}{Ejemplo}
%   Un \textbf{trabajo} puede consistir en 2 requisitos:
%   \begin{enumerate}
%   \item 1 nodo, 512 MB RAM como m�nimo
%   \item 24 nodos, 128 MB como m�nimo.
%   \end{enumerate}
% \end{block}


\end{frame}

\end{document}
