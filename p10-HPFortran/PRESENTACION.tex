\documentclass[10pt]{beamer}

\usetheme{m}

\usepackage{booktabs}
\usepackage[scale=2]{ccicons}

%\usepackage{pgfplots}
%\usepgfplotslibrary{dateplot}

\usepackage[latin1]{inputenc}
\usepackage[spanish]{babel}

\usepackage{listings}
\usepackage{pgfplots}

\usepackage{amsmath}

\title{High Performance Fortran}
\subtitle{}
\date{}
\author{Luis Mar�a Costero Valero\\Jes�s Javier Dom�nech Arellano\\Hristo
  Ivanov Ivanov}
\institute{25 Enero 2016}

\titlegraphic{\hfill\includegraphics[scale=0.2]{HPFF_logo.jpg}}

\definecolor{bgg}{HTML}{FBFBFB}
\def\gcolor{bgg}    % while presenting
%\def\gcolor{black} % while developing

\def\tikzpicdim{
  \draw[step=0.1cm, color=\gcolor] (0,-1) grid (12,7);
  \draw[step=1cm, color=\gcolor] (0,-1) grid (12,7);
}

\let\tikzpicdimlarge\tikzpicdim

\def\myurl{\hfil\penalty 100 \hfilneg \hbox}

\metroset{titleformat=regular}
\metroset{inner/sectiontitleformat=regular}
\metroset{outer/frametitleformat=regular}
\metroset{block=fill}

\lstset{%
  %backgroundcolor=\color{yellow!20},%
    basicstyle=\small\ttfamily,%
    numbers=left, numberstyle=\tiny, stepnumber=1, numbersep=5pt,%
    frame=l%
    }%


\begin{document}

\maketitle

%%======= INDICE ========================================
%\begin{frame}
%  \frametitle{�ndice}
%  \setbeamertemplate{section in toc}[sections numbered]
%  \tableofcontents[hideallsubsections]
%\end{frame}

%\section{Introduction} %=== Esrto genera una p�gina de secci�n.

\begin{frame}[fragile]
  \frametitle{�Qu� es?}
  \emph{High Performance Fortran} es una extensi�n de \emph{Fortran 90} que
  a�ade directivas de distribuci�n de datos. Extiende el paralelismo a nivel de
  datos, permitiendo la distribuci�n de estos datos entre multiples
  procesadores.
\end{frame}

\begin{frame}[fragile]
  \frametitle{Historia.}
  La primera versi�n de \emph{HPF} se publico den 1993 por el \emph{High
  Performance Fortran Forum (HPFF)}, convocado y presidido por Ken Kennedy de
  la Rice University. \\
  Al finalizar la decada de los 90 \emph{HPF} cayo en desuso en vafor a otras
  soluciones como \emph{OpenMP}. Sin embargo \emph{HPF} ha dejado huella
  infuenciado las versiones futuras de \emph{Fortran}.
\end{frame}

\begin{frame}[fragile]
  \frametitle{Directivas de distribuci�n de datos.}
  La directiva \texttt{DISTRIBUTE} especifica la partici�n de lo datos entre
  los procesadores. \\
  La directiva \texttt{ALIGN} fuerza a que los datos correspondientes de dos o
  m�s arrays esten almacenados en un mismo procesador. \\
  La directiva \texttt{PROCESSORS} especifica la configuraci�n de los
  procesadores para el uso con la directiva \texttt{DISTRIBUTE} \\
  La directiva \texttt{TEMLATE} especifica un espacio de indices vacio, para el
  uso con la directiva \texttt{ALIGN}.
\end{frame}

\begin{frame}[fragile]
  \frametitle{Directiva FORALL.}
    \begin{center}
    \texttt{FORALL (triplet, ..., triplet, mask) assignment} \newline
    \end{center}
    donde \texttt{assignment} es una asignaci�n aritm�tica o de punteros y
    \texttt{triplet} tiene el siguiente formato \newline
    \begin{center}
    \texttt{subscript =  lower-bound :  upper-bound :  stride} \newline
    \end{center}
    y especifica un consjunto de indices (siendo \texttt{: stride} opcional)

    \begin{lstlisting}
      FORALL (i=1:m, j=1:n)      X(i,j) = i+j
      FORALL (i=1:n, j=1:n, i<j) Y(i,j) = 0.0
      FORALL (i=1:n)             Z(i,i) = 0.0 
    \end{lstlisting}
\end{frame}

\begin{frame}[fragile]
  \frametitle{Directiva INDEPENDENT.}
    Directiva que prece al \texttt{do-loop} afirmando que las iteraciones del
    bucle pueden realizarse de forma endependiente, en cualquier orden y de
    forma concurrente.
    \begin{lstlisting}
    !HPF$   INDEPENDENT 
            do i=1,n 
              A(Index(i)) = B(i)
            enddo
    \end{lstlisting}
    En el ejemplo de arriba se asume que el array \texttt{Index} no contiene
    elementos repetidos, y que \texttt{A} y \texttt{B} no comparten memoria.
\end{frame}

\begin{frame}[fragile]
  \frametitle{Pure Procedures.}
  \emph{High Performance Fortran} no permite definir funciones puras, sin
  efectos laterales. Estas pueden ser referenciadas en construciones
  \texttt{FORALL} y bucles \texttt{INDEPENDNT do-loop}.
    \begin{lstlisting}
      PURE FUNCTION DOUBLE(X)
      REAL, INTENT(IN) :: X
      DOUBLE = 2 * X
      END FUNCTION DOUBLE
    \end{lstlisting}
\end{frame}

\begin{frame}[fragile]
  \frametitle{Bibliograf�a}
    \begin{thebibliography}{}
        \bibitem{kcc}
            \emph{High Performance Fortran},
            Ian Foster,
            1995,
            \url{http://www.mcs.anl.gov/~itf/dbpp/text/node82.html}.
        \bibitem{kcc}
            \emph{High Performance Fortran Features},
            Compaq Computer Corporation Houston Texas,
            September 1999
            \url{http://h21007.www2.hp.com/portal/download/files/unprot/fortran/docs/lrm/lrm0009.htm}
        \bibitem{kcc}
            \emph{High Performance Fortran},
            Wikipedia,
            2013,
            \url{https://en.wikipedia.org/wiki/High_Performance_Fortran}
        \bibitem{kcc}
            \emph{High Performance Fortran, Official Page},
            Rice University
            2000-2006,
            \url{http://hpff.rice.edu/}
    \end{thebibliography}{}


\end{frame}

\end{document}
