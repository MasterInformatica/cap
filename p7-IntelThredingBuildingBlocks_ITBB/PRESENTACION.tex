\documentclass[10pt]{beamer}

\usetheme{m}

\usepackage{booktabs}
\usepackage[scale=2]{ccicons}

\usepackage{pgfplots}
\usepgfplotslibrary{dateplot}

\usepackage{listings}


\title{Intel Threading Building Blocks}
%\subtitle{}
\date{}%23 Noviembre 2015}
\author{Luis María Costero Valero\\Jesús Javier Domenech Arellano\\Hristo
  Ivanov Ivanov}
\institute{23 Noviembre 2015}


\begin{document}

\maketitle


%%======= INDICE ========================================
\begin{frame}
  \frametitle{Índice}
  \setbeamertemplate{section in toc}[sections numbered]
  \tableofcontents[hideallsubsections]
\end{frame}

\section{Introduction} %=== Esrto genera una página de sección.

\begin{frame}
  \frametitle{Beneficios}
  \begin{itemize}
    \item \textbf{Abstracción de alto nivel}. \emph{TBB} se encarga de
          manejar los \emph{threads}, mientras que el usuario tan solo
          tiene que definir la \textbf{lógica de paralelismo}.
    \item \textbf{\emph{Threading} for performance}. \emph{TBB} está orientado
          a paralelizar tareas de computación intensiva.
    \item \textbf{Compatibilidad con otras librerías de paralelización}.
    \item \textbf{Data-parallel Programming}. Permite a diferentes
          \emph{threads} trabajar sobre partes del mismo conjunto de datos.
    \item \textbf{Generic Programming}. Los algoritmos son implementados
          sobre tipos genéricos, que el usuario especifica en función
          de su necesidad.
  \end{itemize}
\end{frame}

\begin{frame}
  \frametitle{Título Pág 2.}
  Slide 2.
\end{frame}


%===== EJEMPLO =====
\begin{frame}[fragile] % Frame ejemplo 1
  \frametitle{Ejemplo}
  \alert{Problema} de acceso a variable compartida:
  \begin{columns}
    \begin{column}{0.45\textwidth}
      \center Thread A:
      \begin{lstlisting}[language=C++, frame=single]
for( ; ; ){
    // do something ...
    // ...
    work_counter += n;
}
      \end{lstlisting}
    \end{column}
    \begin{column}{0.45\textwidth}
      \center Thread B:
      \begin{lstlisting}[language=C++, frame=single]
for( ; ; ){
    // do something ...
    // ...
    work_counter += n;
}
      \end{lstlisting}
    \end{column}
  \end{columns}
  \alert{Solución} usando pthreads:
  \begin{itemize}
  \item Protejer la variable (Mutex, cerrojos, semáforos, ...)
  \end{itemize}
\end{frame}

\begin{frame}[fragile] % Frame ejemplo 2
  \frametitle{Ejemplo}
  \alert{Solución} usando Intel TBB:

  \begin{lstlisting}[language=C++, frame=single]
tbb::atomic<int> work_counter;
// ...

for( ; ; ){
    // do something ...
    // ...
    work_counter += n;
}
  \end{lstlisting}
  Internamente está implementado:
  \begin{itemize}
  \item work\_counter.fetch\_and\_store
  \item work\_counter.fetch\_and\_add
  \item work\_counter.compare\_and\_swap
  \end{itemize}
\end{frame}

\end{document}
